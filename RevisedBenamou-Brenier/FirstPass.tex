To create a Manim animation for the first section introducing the metric space \(\mathbb{W}_{p}(\Omega)\), we'll structure the content into seven scenes. Each scene will visualize key concepts with equations, diagrams, and animations.

---

### **Scene 1: Introduction to \(\mathcal{P}_{p}(\Omega)\)**
- **Visuals**:
  - A 2D domain \(\Omega\) (e.g., a circle) with a probability measure \(\mu\) represented as a Gaussian distribution.
  - Equations fade in:
    \[
    \mathcal{P}_{p}(\Omega) := \left\{\mu \in \mathcal{M}(\Omega) \,\bigg|\, \mu(\Omega)=1, \, \int_{\Omega}|x|^p \, d\mu(x) < +\infty \right\}
    \]
- **Animations**:
  - Highlight the domain \(\Omega\) and the measure \(\mu\).
  - Emphasize the constraints (total mass = 1, finite \(p\)-th moment).

---

### **Scene 2: Wasserstein Distance \(W_p\)**
- **Visuals**:
  - Two measures \(\mu\) and \(\nu\) (e.g., Gaussian blobs at different positions).
  - Multiple transport plans (arrows between \(\mu\) and \(\nu\)) with varying costs.
  - Optimal transport plan \(\gamma\) highlighted.
- **Equations**:
  \[
  W_p(\mu, \nu) := \left(\min_{\gamma \in \operatorname{ADM}(\mu, \nu)} \int_{\Omega \times \Omega} |x-y|^p \, d\gamma(x,y)\right)^{1/p}
  \]
- **Animations**:
  - Show transport plans as arrows; fade out non-optimal ones.
  - Compute cost for the optimal \(\gamma\) and display \(W_p\).

---

### **Scene 3: \(W_p\) is a Distance (Proposition 1.1)**
- **Visuals**:
  - Text: "Proposition: \(W_p\) satisfies non-negativity, symmetry, and triangle inequality."
  - Three properties displayed with icons:
    - Non-negativity: \(W_p \geq 0\) (scale with zero).
    - Symmetry: \(W_p(\mu, \nu) = W_p(\nu, \mu)\) (swap \(\mu\) and \(\nu\)).
    - Triangle inequality: Path through intermediate measure.
- **Animations**:
  - Fade in each property with illustrative examples.

---

### **Scene 4: Curves on a Metric Space**
- **Visuals**:
  - Abstract metric space \(X\) (grid background) with a curve \(\omega: [0,1] \to X\).
  - Moving point along the curve with parameter \(t\).
- **Equations**:
  \[
  \text{Curve: } \omega: [0,1] \to X, \quad \text{Metric derivative: } |\omega'|(t) = \lim_{s \to t} \frac{d(\omega(s), \omega(t))}{|t-s|}
  \]
- **Animations**:
  - Animate the curve \(\omega(t)\) as a path.
  - Zoom into two points \(\omega(s)\) and \(\omega(t)\), compute their distance, and show the limit.

---

### **Scene 5: Absolutely Continuous Curves**
- **Visuals**:
  - Curve \(\omega(t)\) with a slider for \(t\).
  - Graph of \(g \in L^1([0,1])\) below, integrating over \([s, t]\).
- **Equations**:
  \[
  d(\omega(t), \omega(s)) \leq \int_s^t g(\tau) \, d\tau
  \]
- **Animations**:
  - Highlight the segment \([s, t]\) on the curve and the corresponding integral on the graph.

---

### **Scene 6: Metric Derivative and Length (Theorem 1.3)**
- **Visuals**:
  - Curve \(\omega(t)\) partitioned into intervals \(t_0, t_1, \dots, t_n\).
  - Sum of distances \(\sum d(\omega(t_i), \omega(t_{i+1}))\) vs. integral of \(|\omega'|(t)\).
- **Equations**:
  \[
  \operatorname{length}(\omega) = \sup \left\{\sum d(\omega(t_i), \omega(t_{i+1}))\right\} = \int_0^1 |\omega'|(t) \, dt
  \]
- **Animations**:
  - Show partitions refining; sum approaches integral.
  - Transition from discrete sum to continuous integral.

---

### **Scene 7: Proposition 1.5 (Length Formula)**
- **Visuals**:
  - Side-by-side: A curve with metric derivative \(|\omega'|(t)\) and the integral \(\int |\omega'| \, dt\).
- **Equation**:
  \[
  \operatorname{length}(\omega) = \int_0^1 |\omega'|(t) \, dt
  \]
- **Animations**:
  - Highlight the equivalence between the supremum of sums and the integral.

---

### **Final Scene: Compilation of Key Concepts**
- **Visuals**:
  - Recap: \(\mathcal{P}_p(\Omega)\), \(W_p\), curves, metric derivative, length.
- **Animations**:
  - Fade in all key equations and diagrams sequentially.
  - End with the title \(\mathbb{W}_p(\Omega)\) as a metric space.

---

**Manim Code Snippet (Scene 1 Example):**
```python
from manim import *

class IntroProbabilitySpace(Scene):
    def construct(self):
        # Create domain Omega (circle)
        omega = Circle(radius=2, color=BLUE)
        label_omega = Tex(r"$\Omega$").next_to(omega, DOWN)

        # Create probability measure (Gaussian)
        mu = FunctionGraph(
            lambda x: np.exp(-x**2) * 2,
            color=YELLOW,
            x_range=[-3, 3]
        ).shift(UP * 0.5)

        # Equations
        equation = MathTex(
            r"\mathcal{P}_p(\Omega) := \left\{\mu \in \mathcal{M}(\Omega) \,\bigg|\, \mu(\Omega)=1, \, \int_{\Omega}|x|^p \, d\mu(x) < +\infty \right\}"
        ).scale(0.8).to_edge(UP)

        # Animations
        self.play(Create(omega), Write(label_omega))
        self.play(Create(mu))
        self.wait(1)
        self.play(Write(equation))
        self.wait(2)
```

This structure ensures a step-by-step explanation with visual and analytical components, making abstract concepts in optimal transport accessible.